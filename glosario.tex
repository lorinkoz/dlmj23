\lawchapterf{Glosario}

\begin{description}
    \item[Asambleísta] Persona que forma parte de una asamblea convocada.
    \item[Asignación conferencial] Cuota anual que debe pagar todo miembro de la \LMJ{}, establecida y aprobada por la Asamblea Anual.
    \item[Aval pastoral] Documento emitido y firmado por el Pastor de una iglesia, que certifica su aprobación respecto a los participantes que su iglesia envía a un evento.
    \item[Comité Postulador] Comité que analiza los candidatos y realiza una propuesta única para cada cargo, a ser sometida a la consideración de los electores, los que podrán votar por la propuesta presentada o hacer nuevas propuestas.
    \item[Identidad sexual] Percepción que un individuo tiene sobre su propia sexualidad. Está conformada por tres elementos: la identidad de género, la orientación sexual y el rol de género. La identidad de género es el género con el que el individuo se identifica. La orientación sexual es el género por el cual el individuo siente atracción sexual. El rol de género es la manera en que el individuo se comporta. Creemos firmemente que existe un diseño único para la identidad sexual según el sexo biológico, tal y como lo define la Palabra de Dios en Génesis 1:27.
    \item[Junta Asesora] Junta distrital dirigida por el Superintendente de Distrito y formada por todos los pastores del distrito, más los promotores de todas las organizaciones y demás líderes a nivel distrital.
    \item[Junta Consultiva] Junta conferencial dirigida por el Obispo y formada por todos los Superintendentes de Distrito, los Presidentes Nacionales de todas las organizaciones y demás líderes a nivel nacional, más un representante laico por distrito.
    \item[Junta Local de Trabajo] Junta dirigida por su Presidente y el Pastor de la iglesia, y formada por todos los presidentes de organizaciones y demás cargos electivos.
    \item[Moción] Proposición que se hace o sugiere en una junta que delibera.
    \item[Quórum] Número de personas necesarias para poder tomar ciertos acuerdos.
    \item[Terna] Las tres propuestas que realiza la Ligal Local o la Asamblea Anual para Asesor Local o Asesor Conferencial, respectivamente, las cuales deberán ser ordenadas por número de votos y sometidas a la decisión final del órgano competente (Junta Local o Junta Consultiva).
\end{description}
