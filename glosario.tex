\lawchapterf{Glosario}

\begin{description}
    \item[Asambleísta] Persona que forma parte de una asamblea convocada.
    \item[Comité Postulador] Comité que evalúa los candidatos y realiza una propuesta única para cada cargo, a ser sometida a la consideración de los electores, los que podrán votar por la propuesta realizada o hacer nuevas propuestas.
    \item[Identidad sexual] Percepción que un individuo tiene sobre su propia sexualidad. Está conformada por tres elementos: la identidad de género, la orientación sexual y el rol de género.
    \item[Junta Asesora] Junta distrital dirigida por el Superintendente de Distrito y formada por todos los pastores del distrito, más los promotores de todas las organizaciones y demás líderes a nivel distrital.
    \item[Junta Consultiva] Junta conferencial dirigida por el Obispo y formada por todos los Superintendentes de Distrito, los Presidentes Nacionales de todas las organizaciones y demás líderes a nivel nacional, más un representante laico por distrito.
    \item[Junta Local de Trabajo] Junta dirigida por el Pastor de la iglesia y formada por todos los presidentes de organizaciones y demás líderes locales.
    \item[Moción] Proposición que se hace o sugiere en una junta que delibera.
    \item[Quórum] Número de personas necesarias para poder tomar ciertos acuerdos.
    \item[Terna] Las tres propuestas que realiza la Ligal Local o la Asamblea Anual para Asesor Local o Asesor Conferencial, respectivamente, las cuales deberán ser ordenadas por número de votos y sometidas a la decisión final del órgano competente.
\end{description}
