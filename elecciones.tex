\lawchapter{Elecciones}
\label{elecciones}

\lawsection{Sección Primera: Generalidades}

\article
\label{requisito-cargos}
Para ser elegido en culquier cargo de la \LMJ{} se requiere ser miembro de la iglesia y contar con la aprobación del Pastor, Superintendente u Obispo, a los diferentes niveles (local, distrital y nacional).

\article
\label{requisito-presidente}
Para ser elegido Presidente a todos los niveles se requiere el voto positivo de la mitad más uno de los electores.

\article
\label{cargos-vacantes}
Cuando un cargo de la \LMJ{} queda vacante antes de concluir el período para el cual fue elegido, se podrán utilizar los siguientes métodos para su cubrimiento:
\begin{enumerate}
    \item Elección temporal.
    \item Designación temporal.
\end{enumerate}
El método será seleccionado por el Consejo Local, previa consulta con el Pastor y el Asesor, o por el Consejo Conferencial, previa consulta con el Asesor Conferencial, según el nivel.

\lawsection{Sección Segunda: Nivel Local}

\article
Las elecciones serán anuales, y se realizarán en la Reunión de Planeamiento más próxima a la Asamblea Anual.

\article
El quórum requerido para llevar a cabo las elecciones será la mitad más uno de los miembros de la liga.

\article
Las elecciones serán dirigidas por el Pastor o el Asesor, los cuales escucharán las proposiciones de los miembros de la Liga Local para cada cargo, y las someterán a votación. Aquellas ligas que debido a sus características lo requieran, podrán utilizar un comité postulador como paso previo a las elecciones.

\article
El Asesor será elegido por la Junta Local de Trabajo, a partir de la terna sugerida por el Consejo Local.

\lawsection{Sección Tercera: Nivel Distrital}

\article
Las elecciones del Promotor de Distrito tendrán lugar en la Conferencia de Distrito más próxima a la Conferencia Anual.

\article
\label{direccion-eleccion-promotor}
La elección del Promotor de Distrito será dirigida por otro miembro del Consejo Conferencial de la \LMJ{}.

\article
El Promotor será elegido por los miembros de la \LMJ{} presentes en la Conferencia de Distrito.

\article
El Promotor de Distrito será elegido en consonancia con la Disciplina de la Iglesia Metodista en Cuba, la cual estipula que: La Iglesia sede de la conferencia no podrá tener, en ninguna elección, mayor número de votantes que la iglesia visitante que más delegados con voto lleve, y la cantidad de delegados de la iglesia visitante que cuente con el mayor número se ha de equiparar al de la iglesia que le siga en número de asistencia.

\lawsection{Sección Cuarta: Nivel Nacional}

\article
Las elecciones se realizarán en la Asamblea Anual.

\article
Las elecciones serán dirigidas por el Asesor Conferencial, y utilizarán un comité postulador como paso previo.

\article
El comité postulador estará dirigido por el Asesor Conferencial, y formado por los Promotores de Distrito y un representante adicional, nombrado en reunión con el resto de los delegados de cada distrito.

\article
El comité postulador presentará su candidatura y posteriormente los delegados podrán hacer sus proposiciones libremente.

\article
Los cargos de Presidente y Vicepresidente deberán tener, como mínimo, cuatro años como miembros de la iglesia.

\article
El Asesor será nombrado por la Junta Consultiva, a partir de la terna que a esta eleve la Asamblea Anual de la \LMJ{}.
