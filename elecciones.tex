\lawchapter{Elecciones}
\label{elecciones}

\lawsection{Sección Primera: Nivel Local}

\article
Las elecciones serán anuales, y se realizarán en la Reunión de Planeamiento más próxima a la Asamblea Anual.

\article
El quórum requerido para llevar a cabo las elecciones será la mitad más uno de los miembros de la liga.

\article
Las elecciones serán dirigidas por el Pastor o el Asesor, los cuales escucharán las proposiciones de los miembros de la Liga Local para cada cargo, y las someterán a votación.

\article
Para ser elegido en los cargos de Presidente y Vicepresidente de la \LMJ{} se requiere ser miembro de la iglesia.

\article
Para ser elegido Presidente se requiere el voto positivo de la mitad más uno de los electores.

\article
El Asesor será elegido por la Junta Local de Trabajo, a partir de la terna sugerida por el Consejo Local.

\lawsection{Sección Segunda: Nivel Distrital}

\article
Las elecciones del Promotor de Distrito tendrán lugar en la Conferencia de Distrito más próxima a la Conferencia Anual.

\article
\label{requisitos-promotor}
Para ser elegido Promotor de Distrito se requiere ser miembro de la iglesia.

\article
\label{direccion-eleccion-promotor}
La elección del Promotor de Distrito será dirigida por otro miembro del Consejo Conferencial de la \LMJ{}.

\article
El Promotor será elegido por los miembros de la \LMJ{} presentes en la Conferencia de Distrito, con la aprobación del Superintendente.

\article
El Promotor de Distrito será elegido en consonancia con la Disciplina de la Iglesia Metodista en Cuba, la cual estipula que: La Iglesia sede de la conferencia no podrá tener, en ninguna elección, mayor número de votantes que la iglesia visitante que más delegados con voto lleve, y la cantidad de delegados de la iglesia visitante que cuente con el mayor número se ha de equiparar al de la iglesia que le siga en número de asistencia.

\lawsection{Sección Tercera: Nivel Nacional}

\article
Las elecciones se realizarán en la Asamblea Anual.

\article
El comité postulador estará dirigido por el Asesor Conferencial, y formado por los Promotores de Distrito y un representante adicional, nombrado en reunión con el resto de los delegados de cada distrito.

\article
El comité postulador presentará su candidatura y posteriormente los delegados podrán hacer sus proposiciones libremente.

\article
Los cargos de Presidente y Vicepresidente deberán tener, como mínimo, cuatro años como miembros de la iglesia.

\article
Los cargos de Secretario, Tesorero, Estadístico, Director de Publicidad, Director de Programa y Director del \OOLMJ{} deberán ser miembros de la iglesia.

\article
Para ser elegido Presidente se requiere el voto positivo de la mitad más uno de los electores.

\article
El Asesor será nombrado por la Junta Consultiva, a partir de la terna que a esta eleve la Asamblea Anual de la \LMJ{}.
